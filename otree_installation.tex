% Created 2016-10-18 Di 11:04
\documentclass[11pt]{article}
\usepackage[utf8]{inputenc}
\usepackage[T1]{fontenc}
\usepackage{fixltx2e}
\usepackage{graphicx}
\usepackage{grffile}
\usepackage{longtable}
\usepackage{wrapfig}
\usepackage{rotating}
\usepackage[normalem]{ulem}
\usepackage{amsmath}
\usepackage{textcomp}
\usepackage{amssymb}
\usepackage{capt-of}
\usepackage{hyperref}
\usepackage[bitstream-charter]{mathdesign}   % fontstyle
\usepackage{xcolor}
\setlength\parskip{1ex}
\setlength\parindent{0}
\usepackage{a4wide}
\newcommand{\mscode}[1]{\colorbox{blue!40!black}{\footnotesize\color{white}\tt #1}}
\newcommand{\maccode}[1]{{\footnotesize\color{black}\tt #1}}
\newcommand{\lincode}[1]{\colorbox{black}{\footnotesize\color{green}\tt #1}}
\author{Felix Albrecht}
\date{\today}
\title{Otree Installation}
\hypersetup{
 pdfauthor={Felix Albrecht},
 pdftitle={Otree Installation},
 pdfkeywords={},
 pdfsubject={},
 pdfcreator={Emacs 24.5.1 (Org mode 8.3.6)}, 
 pdflang={English}}
\begin{document}

\maketitle
\tableofcontents



\section{Windows}
\label{sec:orgheadline10}

\subsection{Required Software}
\label{sec:orgheadline5}

\subsubsection{Python (= 3.5.x)}
\label{sec:orgheadline3}

\begin{enumerate}
\item Quick Installation
\label{sec:orgheadline1}

I recommend installing the \href{https://www.continuum.io/downloads}{Anaconda Python 3.5 Environment}. It's an easy to install version of python that comes with numerous bells and whistles.
You find the link to the website here: \url{https://www.continuum.io/downloads}

\item Advanced Installation
\label{sec:orgheadline2}

Alternatively you can install \href{https://www.python.org/downloads/release/python-352/}{raw python}. However, this requires you to manually set the system path variables.

\textbf{Note:} A Python 2.7.x installation is insufficient. Not all current features of otree work with Python 2.7.x.

\textbf{Note 2:} Do NOT install the recently made available Python 3.6 beta. Compatibility to the current version of otree is not assured.
\end{enumerate}

\subsubsection{otree (> 0.9.5)}
\label{sec:orgheadline4}

Access the Windows powershell. Enter \mscode{pip3 install -U otree-core}. (Userspace installation of otree) Pip3 will download all required packages and install them into your system.

\subsection{Running otree}
\label{sec:orgheadline6}

Open the powershell and navigate to the directory \mscode{cd \textasciitilde\textbackslash\ldots} where you want your otree project folder to be located.

Generate you first otree project with \mscode{otree startproject otree\_project}.

\textbf{Note:} You might experience the following error:

\mscode{error: Microsoft Visual C++ 9.0 is required (Unable to find vcvarsall.bat). ...}

If you do, you need to install the \href{http://go.microsoft.com/fwlink/?LinkId=691126}{Visual C++ Build Tools}.


\subsection{Recommended Software}
\label{sec:orgheadline9}

\subsubsection{git (Version control and collaboration)}
\label{sec:orgheadline7}
I recommend using \href{https://tortoisegit.org/}{Tortoise-Git}. It offers easy Windows Shell integration and Right-Click Menu operations on Git Repository folders.

If you like a graphical interface in addition to the terminal, the \href{https://www.sourcetreeapp.com/}{Sourcetree App} is supposed to be easy to use.

\subsubsection{Pycharm (Python / Django IDE)}
\label{sec:orgheadline8}
The texteditor \href{http://www.jetbrains.com/pycharm/}{Pycharm} is optimized for Python development and offers many features that make Python editing and project management much easier and quicker. It offers direct integration with the above mentioned Git.


\newpage

\section{Mac OSX}
\label{sec:orgheadline20}
\subsection{Required Software}
\label{sec:orgheadline15}

\subsubsection{Python (= 3.5.x)}
\label{sec:orgheadline13}

\begin{enumerate}
\item Quick Installation
\label{sec:orgheadline11}

I recommend installing the \href{https://www.continuum.io/downloads}{Anaconda Python 3.5 Environment}. It's an easy to install version of python that comes with numerous bells and whistles.
You find the link to the website here: \url{https://www.continuum.io/downloads}

\item Advanced Installation
\label{sec:orgheadline12}

Installing Python via \emph{Homebrew}.

In a terminal window input: \maccode{xcode-select --install}

When propted for the installation parameters input:

\maccode{ruby -e "\$(curl -fsSL https://raw.githubusercontent.com/Homebrew/install/master/install)"}

Set the PATH variable for the homebrew environment:

\maccode{echo "export PATH=/usr/local/bin:/usr/local/sbin:\textbackslash\$PATH" >> \textasciitilde/.bash\_profile}

Reload the you terminal config:

\maccode{source ~/.bash_profile}

Install Python 3:

\maccode{brew install python3}

You can check if everything went well with:

\maccode{pip3 -V}
\end{enumerate}

\subsubsection{otree (> 0.9.5)}
\label{sec:orgheadline14}

In the terminal window enter: 

\maccode{pip3 install -U otree-core}. 

(Userspace installation of otree) Pip3 will download all required packages and install them into your system.

\subsection{Running otree}
\label{sec:orgheadline16}

Open a new terminal window and navigate to the directory \maccode{cd \textasciitilde/\ldots} where you want your otree project folder to be located.

Generate your first otree project with \maccode{otree startproject otree\_project} .

\subsection{Recommended Software}
\label{sec:orgheadline19}

\subsubsection{git (Version control and collaboration)}
\label{sec:orgheadline17}

You can download git from here \url{https://git-scm.com/download/mac} .

If you like a graphical interface in addition to the terminal, the \href{https://www.sourcetreeapp.com/}{Sourcetree App} is supposed to be easy to use. But note this is just a frontend you will still need to install git.

\subsubsection{Pycharm (Python / Django IDE)}
\label{sec:orgheadline18}
The texteditor \href{http://www.jetbrains.com/pycharm/}{Pycharm} is optimized for Python development and offers many features that make Python editing and project management much easier and quicker. It offers direct integration with the above mentioned Git.


\newpage

\section{Linux (Ubuntu 16.04)}
\label{sec:orgheadline28}

\subsection{Required Software}
\label{sec:orgheadline23}

\subsubsection{Python 3.x}
\label{sec:orgheadline21}

Ubuntu 16.04 by default come with an installation of Python 3.5. But you will need to install \texttt{pip3}.
I recommend installing everything in one go. 
In the console enter:

\lincode{sudo apt install python3-pip git}

As Ubuntu comes with Python 2.7.x and 3.5.x the default is mapped to 2.7.x. To make handling python and pip quicker you can assign aliases to the commands.

At the end of \texttt{\textasciitilde/.bashrc} insert:

\lincode{alias python=python3}\\
\lincode{alias pip=pip3}

\subsubsection{otree (> 0.9.5)}
\label{sec:orgheadline22}

In the console enter: 

\lincode{pip3 install -U otree-core}. 

(Userspace installation of otree) Pip3 will download all required packages and install them into your system.
For a systemwide installation do:

\lincode{sudo pip3 install otree-core}. 

I recommend doing a systemwide installation as it allows you to run your own production server as non-root user.

\subsection{Running otree}
\label{sec:orgheadline24}

Open a new terminal window and navigate to the directory \lincode{cd \textasciitilde/\ldots} where you want your otree project folder to be located.

Generate your first otree project with \lincode{otree startproject otree\_project} .

\subsection{Recommended Software}
\label{sec:orgheadline27}

\subsubsection{Pycharm (Python / Django IDE)}
\label{sec:orgheadline25}

You can install Pycharm from a launchpad repository.

\lincode{sudo add-apt-repository ppa:mystic-mirage\/pycharm}\\
\lincode{sudo apt update}\\
\lincode{sudo apt install pycharm}\\

\subsubsection{Personal recommendation}
\label{sec:orgheadline26}

For efficient editing and serverside console editing try learning Vim. Every Linux based server offers a version of Vi(m) for quick file editing. I personally use Emacs(Spacemacs) with a Vim runtime environment.

\section{Otree / Web-development Links}
\label{sec:orgheadline29}

\begin{itemize}
\item \href{http://otree.readthedocs.io}{otree online documentation and examples}
\item \href{https://groups.google.com/forum/#!forum/otree}{otree Mailing list}
\begin{itemize}
\item contains many examples and solutions for problems
\item very active list
\item requires subscription to list
\end{itemize}
\item \href{http://www.w3schools.com/js/}{W3School Javascript Documentation}
\begin{itemize}
\item otree is web technology based. Many animations and client side feature require javascript. This is the most comprehensive online documentation.
\end{itemize}
\item \href{http://jquery.com/}{JQuery - Advanced Javascript animations}
\begin{itemize}
\item otree come preloadeds with JQuery to support quick and easy javascript animations.
\end{itemize}
\item \href{http://www.w3schools.com/bootstrap/}{Bootstrap - Advanced CSS}
\begin{itemize}
\item otree come preloaded with the bootstrap CSS framework for beautiful buttons and web-design elements.
\end{itemize}
\item \href{http://www.learnpython.org/}{Python coding tutorial}
\begin{itemize}
\item otree is based on the \href{https://www.djangoproject.com/}{Django Web-Framework} and relies heavily on python. This is goo tutorial to get started with python3.
\item Alternatively I recommend the \href{https://www.codecademy.com/}{Codecademy Python Course}.
\end{itemize}
\end{itemize}



\vfill
Materials available on: \url{https://github.com/cataclysmic/otree-intro}
\end{document}